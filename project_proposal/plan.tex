\section{Project Plan / Milestones}

\subsection{Timeline}

\begin{tabular}{r|l}
	Milestone & Date \\\hline\\
	Proposal Due & March 30\textsuperscript{th} \\
	Familiarise with JIF & April 13\textsuperscript{th} \\
	Develop Evaluation Criteria & May 4\textsuperscript{th} \\
	Produce Email Client Specification & May 11\textsuperscript{th} \\
	Progress Seminar & May 15\textsuperscript{th}-19\textsuperscript{th} \\
	Complete Java Implementation & June 29\textsuperscript{th} \\
	Complete JIF Implementation & August 11\textsuperscript{th} \\
	Determine Evaluation Results & September 14\textsuperscript{th} \\
	Complete Evaluation Write-up & October 6\textsuperscript{th}\\
	Poster and Demonstration & October 20\textsuperscript{th} \\
	Final Thesis & November 6\textsuperscript{th} \\
\end{tabular}

\subsection{Project Milestones}

\subsubsection{Familiarise with JIF / Information Flow languages}

Before concrete evaluation criteria and work on the motivating example can begin, familiarisation with JIF and Information Flow security models more generally will be required. This process will involve reading documentation, setting up a work environment and writing small code examples in order to try and gain a working knowledge of JIF, as well to try and produce examples which illustrate particular security properties for later use in demonstration and the Progress Seminar.

The primary resource for this task will be the JIF documentation and toolset (e.g. the Eclipse-based JIF IDE), along with more theoretical information from the literature on JIF and static Information Flow security more generally.

It is expected that this first phase will take about three weeks; ideally by the mid-semester break in mid-April the JIF toolset and workflow should be set up, and enough of an understanding should have been gained to allow for criteria to begin to be developed.

\subsubsection{Develop Evaluation Criteria}

The criteria used to evaluate the Information Flow secure models (as well as the standard Java security model) will need to be developed by considering the core purpose behind the evaluation: to determine \textit{what} benefits Information Flow secure models can bring, \textit{why} they are not being used widely in production applications, and \textit{how} they could be made more accessible for broader use.

Developing these criteria will require the use of resources primarily from the security literature. The Information Flow literature and the specifications for the Java security model will aid in developing criteria that can accurately evaluate the security guarantees and properties that a particular model provides. For the Java security model, other literature analysing the model, as well as discussions of known exploits and weaknesses will be of use. The development of criteria around usability and programmer burden will derive from literature on the usability of the Java model. Resources on the usability and practicality of Information Flow models are less well developed, and so this criteria will be developed more as a result of the familiarisation with JIF.

The criteria will continue to be refined throughout the project, and it is expected that they will be significantly updated through the process of developing the secure email client as a motivating example in Java and JIF, as a deeper understanding of exactly how these systems work in practice is gained.

It is expected that a base set of criteria will be developed by the beginning of May, leaving time for familiarising with JIF. Though the criteria will almost certainly evolve later into the project, it is desirable for a clear and easily articulated basis for them be ready to be presented at the Progress Seminar in mid-May.

\subsubsection{Produce Secure Email Client Specification}

Concurrently with the development of the evaluation criteria, a detailed design for the motivating example of the secure email client will be developed. This document should in broad terms describe the required functionality of the email client, illustrating its general design, as well as outlining potential challenges for implementation. In addition, the specification should clearly demarcate which functionality is core to the example and which will be desirable extension that are nonetheless not required should time be constrained.

The development of this specification will depend on documentation around email protocols, as well as the documentation on developing with the Java security model and JIF in order to estimate the challenges of implementation.

The email client specification should be clearly laid out by early May, within a similar timeframe to the evaluation criteria, so that at the very least a broad outline can be presented at the Progress Seminar; ideally the specifics should be clearly defined by this stage as well.

\subsubsection{Develop Java Implementation of Secure Email Client}

Once the specification is produced, it will be implemented in Java, with the SecurityManager enabled and a security policy designed to try and prevent unauthorised information flows. Notes will be kept about the development process to aid with later evaluation.

The implementation will depend upon the Java Development Kit, an IDE and toolset, Java documentation, and upon documentation of the Java Security Model (including \textit{Inside Java 2 Platform Security} by Gong and Ellison \cite{gong2003javasecurity}).

While some implementation may begin while the specification is being finalised, it is expected to begin in earnest at around the time of the Progress Seminar, to be completed by the end of June.

\subsubsection{Develop JIF Implementation of Secure Email Client}

The JIF implementation will be built to the same specification as the Java implementation. It is expected that the Java version will form the base, with the JIF implementation sharing a large amount of source code. Nonetheless it is crucial that the security typing system be used to implement meaningful Information Flow controls that provide the guarantees listed in the specification. As with the Java implementation, notes for the further evaluation will be kept.

This implementation will depend on resources including the JIF language documentation and the JIF Eclipse plugin as well as the Information Flow literature to ensure that the security policy defined by the program provides the necessary security guarantees.

The JIF implementation will begin after the Java implementation is complete, and is expected to be completed by mid-August, leaving a month and a half for full implementation.

\subsubsection{Determine Evaluation Results}

The evaluation criteria, which will have been refined during the implementation phase, will be used to perform the evaluation of the security models of each implementation, based on the security guarantees each provides, as well as observations from development on the programmer burden that development using each model caused. The result of this phase will be a set of notes which will eventually form the basis of the project's conclusions.

This resources that this phase will primarily make use of are the implementations from the earlier phases, notes from their development phase detailing developer burden, as well as literature around the security guarantees of each model.

This evaluation has been allocated a month up until the middle of September in order to allow for the collation of notes and the detailed and considered application of the criteria. This process will blur somewhat with the subsequent write-up of the evaluation: it is likely that some elements will still be under consideration when the early portions of the write-up are being worked on.

\subsubsection{Perform Evaluation Write-up}

The completed write-up and table detailing the application of the criteria will be produced following the determination of the evaluation's results, though some aspects of the write-up may happen concurrently with the evaluation itself. This period will also be used to refine and write up other aspects of the final thesis document,

This evaluation write-up will make use of the criteria and results from the previous stage, as well as literature used to refine the content of the evaluation and other aspects of the final thesis.

The final write-up, along with a mostly complete thesis document, should be completed by early October, to allow time for preparation of the Poster and for the Project Demonstration, as well as fine-tuning of the thesis document in preparation for submission before the start of November.

\pagebreak

\subsection{Extension Goals and Scope Considerations}
	
	\subsubsection{Scope of Motivating Example (Email Client)}
	
	The scope of functionality implemented in the Java and JIF versions of the email client is one area which presents opportunity to pull back or extend depending on time and resource constraints. The specification will be designed with this in mind, so that if during development it is necessary too reduce scope, no re-architecture is necessary, and if development proceeds more quickly than anticipated then extended features can be added without requiring extra time to design and specify them.
	
	\subsubsection{Run-Time Enforcement Systems}
	
	Run-time enforced Information Flow security models are newer and less commonly available than statically enforced models. In addition, most run-time enforced models in the literature focus on web languages (particularly JavaScript) rather than statically typed languages such as Java. While it is intended that run-tiem enforced systems be considered with respect to the evaluation, developing a version of the email client in a run-time enforced system is less feasible. Hence, more detailed use of run-time enforced systems to aid in the evaluation may be considered as an extension goal which will be pursued to the extent that time and resources allow, but it is somewhat secondary to the primary goal of the evaluation.
	
	\subsubsection{Paragon Implementation}
	
	Paragon provides statically enforced Information Flow security similar to that of JIF, but as a newer language it provides a number of new concepts (such as selective declassification based on `flow locks'). It is expected that developing the email client in Paragon, in addition to Java and JIF would be an inefficient use of resources, but if time and resources allow, examining Paragon in greater depth through the use of examples is a potential extension for the project, in order to determine whether Paragon's extra concepts improve the security guarantees or reduce programmer burden when compared to more traditional mechanisms like JIF.
	
	\subsubsection{Security Exploits Examination}
	
	It may be useful to the broad scope of the evaluation to consider existing exploits which make use of flaws in the standard Java security model, in order to determine how useful the security guarantees of Information Flow models are in practice. However, reproducing and understanding specific exploits is likely to be difficult and time consuming, and linking an exploit to a specific failure of the Java security model requires a very high level of understanding. As a result, examination of specific exploits is not considered core to the scope of the project, but may be considered if the implementation is completed more quickly than expected.