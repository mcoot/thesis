\section{Thesis Topic}

\subsection{Topic}

The topic for the thesis is ``Analysis of the Security Properties of Java Programs", and more specifically the Java security model, alternate security models for Java, and their relation to principles of Information Flow. 

\subsection{Project Goals}

The overall goal of the project is to perform an evaluation of the Information Flow-based security models and implementations (such as the Java Information Flow (JIF) and Paragon language extensions \cite{pullicino2014jif}\cite{broberg2013paragon}) developed in academia, which are not widely used by production Java applications. The evaluation will compare these against existing Access Control-based security models in terms of the confidentiality guarantees they provide, as well as the burden working with them places upon developers, with the aim of determining why Information Flow models are not used outside of academia, and identifying the properties that a `production-viable' Information Flow security implementation should have.

\subsection{Project Outline}

In order to perform this evaluation, a set of criteria will be developed concerning the confidentiality guarantees made by a model, its effectiveness in preventing or mitigating security exploits, as well as the requirements and restrictions placed upon programmers seeking to make use of the model. These requirements will be formulated and exemplified by a motivating case example, an Information Flow-secure email client. The email client will be written in both pure Java (using the Access Control-based Java security model) and in JIF, with the differences in security guarantees and programming difficulty forming a large part of the evaluation.

Ultimately the outcome this project seeks to produce is a thorough demonstration and evaluation of the state of Information Flow security models for Java, their advantages and disadvantages, and why Information Flow-based models have not seen more widespread adoption within industry.

\subsection{Project Relevance}

This project has significant relevance to the current state of software development. Java is widely used throughout academia and industry, and has been a large target for attacks and exploits aimed at violating the confidentiality of information.

In addition, the security models most commonly associated with Java provide very few guarantees around Information Flow, and  information is often leaked to a lower level of confidentiality as a result of limitation in Access Control Mechanisms, creating information flows which were not intended by the program's creator. On the other hand, languages and security models which provide strong Information Flow guarantees exist, but are rarely used outside of academia. By evaluating these Information Flow models, this project aims to provide insight into the reasons behind this divide.