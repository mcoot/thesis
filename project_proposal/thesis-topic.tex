\section{Thesis Topic}

\subsection{Topic}

The topic for the thesis is ``Analysis of the Security Properties of Java Programs", and more specifically the Java security model, alternate security models for Java, and their relation to principles of Access Control and Information Flow. 

\subsection{Project Goals}

The overall goal of the project is to perform an evaluation of existing, widely used Access Control-based security models in Java (such as the Java Security Manager, and OSGi's security layer (REFERENCE)) as well as Information Flow-based security models developed in academia which are not widely used by production Java applications (such as the Java Information Flow (JIF) and Paragon language extensions (REFERENCE)).

In order to perform this evaluation, a set of criteria will be developed concerning the confidentiality guarantees made by a model, its real world effectiveness in preventing or mitigating security exploits, as well as the requirements and restrictions placed upon programmers seeking to make use of the model. These criteria will be applied to a number of security models based on Access Control and Information Flow methodologies with examples and case studies used to illustrate the strengths and weaknesses of each model in terms of a particular criterion.

Ultimately the outcome this project seeks to produce is a thorough evaluation and taxonomy of the state of Access Control and Information Flow security models for Java, how they differ and the advantages and disadvantages of each, why Information Flow-based models have not seen more widespread adoption within industry.

\subsection{Project Relevance}

This project has significant relevance to the current state of software development, as Java is widely used throughout academia and industry, and has been a large target for attacks and exploits of the confidentiality of data. In addition, the security models most commonly associated with Java provide very few guarantees around Information Flow, and exploits often make use of failures of Access Control mechanisms in order to create information flows that were not intended by the program's creator. On the other hand, languages and security models which provide strong Information Flow guarantees exist, but are rarely used outside of academia. By evaluating both, this project aims to provide insight into the reasons behind this divide.