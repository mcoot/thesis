\section{Risk Assessment}

\subsection{OHS Statement}

This thesis project is not assigned to any particular laboratory. As a result, work will only be performed in low risk laboratory environments (i.e. computer labs), within the UQ OHS guidelines.

\subsection{Electrical Risk}

As the computer labs contain electrical equipment, there is a risk of electrical failure resulting in injury to persons or equipment. Though such a failure could be high in severity, it is extremely low in likelihood. To mitigate it, devices used must be used in a safe manner with good quality electrical cabling.

\subsection{Strain / Injury Risk}

Extended use of computers can result in injury caused by repetitive action and poor posture. The severity of this risk is generally low, but it is quite likely to occur without proper mitigation. As a result, work will be performed making use of a comfortable computer setup with breaks taken regularly to ensure physical health.

\subsection{Data Loss}

As virtually all project information and data is stored electronically, data loss is a significant risk. Such data loss could be caused by hardware failure, physical loss or theft, or unavailability of online cloud service. The potential impact of this risk is very high (it could result in the loss of all project data), and the likelihood is moderate.

To mitigate this risk, all project documents (e.g. this proposal, the thesis write-up, seminar slides, code) will be stored in git version control, with a repository hosted (privately) on GitHub, and local repositories on a number of isolated personal machines, as well as a local repository on the UQ student server (moss). This will ensure that there are physically isolated copies of all the information, and in addition accidental deletion of important data is prevented since the entire change history is available. Other information, such as EndNote citation libraries, will be kept both as libraries on multiple physical machines and synced to an EndNote web account.

\subsection{Risk of Delays in Implementation}

JIF and security typed languages generally are not commonly used outside of academia, and there is a limited amount of documentation and example code available. Hence, there is a risk that familiarising with the language and developing the email client in it will take more time and resources than anticipated. This risk potentially has a moderate to high impact, since delayed development will impact the final evaluation of each model. Its likelihood is moderate; while it is likely that some difficulties will be encountered, it is less likely that issues causing long-term delay will be encountered.

The primary means of mitigation for this risk is the project plan's inclusion of scoping and extension considerations, so that certain aspects of the project can be scoped back to deal with unforeseen delays, or extended as time allows.