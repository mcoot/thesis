\section{Case Study 3: Calendar Scheduler}

\textbf{Explanation}

Users have calendars, which in turn have events. Events have a beginning and start time, a title and a list of attending users (users must attend events on their calendars).

The application has multiple users, and allows users to schedule a time for a new meeting when all users are free. Hence, the times of meetings are not confidential. The titles however are restricted to be visible only to attendees of that meeting.

\textbf{Security properties:}

Some properties of each event are confidential but not others.

The confidentiality of events is bound to their list of attendees (i.e. quantified over some list)

\textbf{Key points:}

Both Paragon and JIF cannot encode the required policies.

JIF fails entirely since lists are a dynamic construct and JIF cannot handle dynamic policies.

Paragon's policy language can encode the idea of the dynamic construct but annotations don't allow universal quantification -- hence the language can express the idea of a singular `owner' of a meeting, but not a list of attendees.

Paragon's compiler cannot however statically determine that different instances are different actors in this case, even in the single owner example.

JIF and Paragon can express the `owner' policy in a way that will (to some degree) work, but it relies on using principals/actors as type parameters.

\textbf{Toy examples:}

Demonstration of failure to determine actors (Paragon)

Simple example where quantification is required

\newpage

\subsection{Overview}

The final case study is an implementation of a multi-user `calendar scheduler' application. The application has multiple users, each with their own calendar. A calendar consists of a list of events; each event has a title, and is associated with a day of the week and starting and finishing time. In addition, each event has a list of users who are attending the event.

The application allows users to schedule new events so as to avoid conflicting with existing events. Each event's details are secret -- only the attendees may know the title of the meeting. However, the times of the meetings need to be available to all users in order to schedule new meetings.

\subsection{Key Security Properties}

On the surface the security property here may seem to lack complexity: as with the second case study we have a data structure in the form of an event which contains some high confidentiality data and some low confidentiality data. Unlike that case, the data is not time-variant but instead dependent on the involvement of particular users -- which seems like a natural fit for the principals of JIF.

However, there are a number of key differences that make this policy complex.

\subsection{Implementation Structure}

\subsection{JIF Implementation}

\subsection{Paragon Implementation}

\subsection{Conclusion}