\section{Case Study 3: Calendar Scheduler}

\textbf{Explanation}

Users have calendars, which in turn have events. Events have a beginning and start time, a title and a list of attending users (users must attend events on their calendars).

The application has multiple users, and allows users to schedule a time for a new meeting when all users are free. Hence, the times of meetings are not confidential. The titles however are restricted to be visible only to attendees of that meeting.

\textbf{Security properties:}

Some properties of each event are confidential but not others.

The confidentiality of events is bound to their list of attendees (i.e. quantified over some list)

\textbf{Key points:}

Both Paragon and JIF cannot encode the required policies.

JIF fails entirely since lists are a dynamic construct and JIF cannot handle dynamic policies.

Paragon's policy language can encode the idea of the dynamic construct but annotations don't allow universal quantification -- hence the language can express the idea of a singular `owner' of a meeting, but not a list of attendees.

Paragon's compiler cannot however statically determine that different instances are different actors in this case, even in the single owner example.

JIF and Paragon can express the `owner' policy in a way that will (to some degree) work, but it relies on using principals/actors as type parameters.

\textbf{Toy examples:}

Demonstration of failure to determine actors (Paragon)

Simple example where quantification is required

\newpage

\subsection{Overview}

\subsection{Key Security Properties}

\subsection{Implementation Structure}

\subsection{JIF Implementation}

\subsection{Paragon Implementation}

\subsection{Conclusion}