\section{Case Study 1: Battleships}

\textbf{Explanation:}

`Battleships' game in which two players have a secret board with ships at various locations. The opposing player may not learn the locations of the enemy ships, except through querying individual coordinates on their turn to test whether that is a battleship position.

\textbf{Security properties:}

Two instances of the Player class which are `mutually distrustful'.

Confidential information must be allowed to be released under specific circumstances (through querying).

\textbf{Key points:}

Modelling a principal-per-player fits well into JIF's policy model.

Fits within JIF's declassification framework -- secret data can be released, but only through specific `escaping' channels.

Paragon also fits well; doesn't need to concretely model actors at all (for confidentiality at least).

Makes good use of Paragon's `escaping block' construct.

Overall: a fairly trivial example which allows for easy expression in both languages. Paragon's expressivity allows for much less work/burden than JIF.

\textbf{Toy examples:}

JIF declassification procedure

Paragon declassification block

\newpage

\subsection{Overview}

The first case study is that of an implementation of the well known board game Battleships. The premise of this game is that two opposing players each have a separate `board', which is a grid of (x, y) coordinates. Each player places a set number of ships on the board, which cover some number of contiguous coordinates either horizontally or vertically. A player is not allowed to learn the layout of the other's board, but from here players take it in turns to guess (or `query') coordinates, and the other player must reveal whether or not the guessed coordinate contains a ship -- if so, that ship is destroyed. The game continues until one player has no remaining ships on the board.

Essentially, this is an incomplete information game: the rules require that each player cannot know the layout of the opponent's board, but they must trust the integrity of the information they receive through querying (i.e. it is not permissible to lie about whether a queried position contains a ship).

\newpage

\subsection{Key Security Properties}

There are two facets to the security policy represented in this case study: the confidentiality and the integrity of each player's board.

\subsubsection{Confidentiality of the Board}

The confidentiality policy is simply stated: a player's board is secret, and information about the positions of ships on the board may flow to the player owning the board and no-one else. The one exception to this is the querying process: an opponent may, on their `turn' in the game, query a coordinate and learn whether or not a ship overlaps the given point.

This exception represents a `declassification' of some information regarding the location of ships on the board (after all, the opponent could gain perfect information by simply querying every point on the board), but it restricts the \textit{quantity} of information which is leaked to one coordinate and hence at most one ship per query, and it ensures that the information may be declassified only through this method and no other.

\subsubsection{Integrity of the Board}

A corollary to the above confidentiality policy is that, in order for the game to proceed according to the rules, the information declassified by a query must be accurate to the state of the queried player's board. Thus, the player must `trust' their opponent to reveal the truth about a queried coordinate.

\subsection{Implementation Structure}\label{battleships_impstructure}

Unlike the other case studies presented here, a JIF implementation of Battleships already exists -- it comes standard with a download of the compiler \cite{jifwebsite}. This implementation not only provides the confidentiality policy described above, but also the integrity policy. Though this is relevant to the overall problem, expression of confidentiality policies is the main focus of this comparison and so the integrity policy is not examined in depth.

In order to provide an effective comparison, the Paragon implementation was developed using the same basic structure as the JIF implementation -- this pattern is repeated for the other case studies, where a common code structure is used for both JIF and Paragon implementations.

This structure is uses the following Java / JIF / Paragon classes:

\begin{itemize}
	\item \mono{Coordinate}: a simple immutable representation of an (x, y) pair
	
	\begin{itemize}
		\item Polymorphic with respect to security policy (i.e. it has no inherent policy, but a policy may be placed upon a coordinate)
	\end{itemize}
	
	\item \mono{Ship}: an immutable single ship in the game
	
	\begin{itemize}
		\item Has a length, a Coordinate for the bottom left point, and a flag indicating whether the ship is aligned horizontally or vertically
		
		\item Provides methods to check whether a given Coordinate is covered by the ship, and whether another given ship intersects this one
		
		\item Polymorphic with respect to security policy
	\end{itemize}
	
	\item \mono{Board}: a board of coordinates containing a list of Ships
	
	\begin{itemize}

		\item Provides a method to test whether a given Coordinate contains a Ship
		
		\item Polymorphic in terms of policy: all ships take on the Board's policy
	\end{itemize}
	
	\item \mono{Player}: a player in the game, with their own secret Board
	
	\begin{itemize}
		\item On creation, the player initialises its board with ship positions
		\item Provides methods to generate new queries, and to receive and answer queries from the opponent
		\item Defines the policy on its board: no-one but the player may view its board, except through its \mono{processQuery} method (which declassifies information about the queried point)
	\end{itemize}
	
	\item \mono{BattleShip}: coordinates the gameplay between two players
	
	\begin{itemize}
		\item Provides the method which runs the game logic
	\end{itemize}
	
	\item \mono{Main}: provides the program's main method
\end{itemize}

\newpage

\subsection{JIF Implementation}

\subsubsection{Impact of Integrity Policy}

As per the \nameref{battleships_impstructure} section, the existing JIF implementation, developed by the creators of JIF themselves, includes an implementation of not only the game's confidentiality policy but also its integrity policy.

The inclusion of this policy significantly increases the complexity of the code: it requires the addition of a number of methods to allow for the `endorsement' of information to a higher level of integrity, with method signatures such as the following from the \mono{Board} class:

\begin{minted}{Java}
public Board[{p->*; p<-* meet o<-*}]{p<-* meet o<-*} 
					endorseBoard{p<-* meet o<-*}
			(principal{p<-* meet o<-*} p, principal{p<-* meet o<-*} o)
where {L} equiv {p->*; p<-*}, caller(p,o) {
\end{minted}

These method signatures are not particularly readable, and when combined as in other areas with JIF's confidentiality labeling system, it can become very difficult to discern even what the return type and parameters of a method mean.

The integrity policy and its implications are not considered more generally here -- for the purposes of this comparison, it is assumed that preventing the downward flow of information is the goal of developing with a security-type system, as opposed to preventing the upward flow of `trust' or integrity.

\subsubsection{Confidentiality Policy}

In contrast with the integrity policy, the implementation of the confidentiality of ship locations is relatively simple.

 The \mono{Player} class has two principal type parameters \mono{P} and \mono{O}, specifying the principals acting as this player and their opponent. The \mono{Board} class takes one label type parameter, specifying the policy on its ships, and hence each Player's Board is parametrised by the policy label \jiflabel{P->*; P<-* meet O<-*}. The confidentiality portion of this (\mono{P->*}) simply states that the Board belongs to \mono{P}, and \mono{P} allows this information to be viewed only by the top principal.
 
 This policy prevents the opposing player (principal \mono{O}) from accessing the information. Only through a declassification statement may information about the state of the board flow to another principal.
 
 The \mono{processQuery} method is the only method which allows for this, through the following two lines (which run after various input checking conditions):
 
 \begin{minted}{Java}
 // find the result.
 boolean result = brd.testPosition(query, new label {P<-* meet O<-*});
 // declassify the result
 return declassify(result, {P->*;P<-* meet O<-*} to {P<-* meet O<-*});
 \end{minted}
 
 The result is calculated by querying the board. The \mono{result} variable is inferred by the type system to be of confidentiality policy \jiflabel{P->*}, and so the \mono{declassify} statement is used to declassify it to the empty confidentiality policy \jiflabel{} (equivalent to \jiflabel{\_->\_}). Hence, the return value of this method may flow to any principal.
 
 The basis of this policy is quite simple, though it does introduce annotation burden throughout the application. JIF's policy model is able to quite effectively represent it, as the `players' intuitively align with statically determined principals, and the query construct is naturally represented with a declassification.

\subsubsection{Boilerplate \& General Details}

In addition to the annotations and additional code required to define the security policy, JIF's design requires further `boilerplate' which is necessary for any program that takes input or produces output. In the case of Battleship, the program outputs information to standard out, which is external to the program and hence lacks confidentiality controls. The boilerplate code (from the Battleship example) illustrating this is as follows:

\begin{minted}{Java}
PrintStream[{}] out = null;
try {
	Runtime[p] runtime = Runtime[p].getRuntime();
	out = runtime==null?null:runtime.stdout(new label {});
}
catch (SecurityException e) {
	// just let out be null.
}

// the PrintStream needs to be labeled {Alice<-* meet Bob<-*}, which
// requires a declassification and an endorsement.
PrintStream[{}] out1 = endorse(out, 
		   				{p->*} to {p->*;Alice<-* meet Bob<-* meet p<-*});
PrintStream[{}] out2 = declassify(out1, {Alice<-* meet Bob<-*});
\end{minted}

In short, this code represents the following steps:

\begin{enumerate}
	\item Get a reference to the JIF \mono{Runtime} object belonging to the program's calling principal (\mono{p})
	
	\item Attempt to get a reference to an unclassified standard out stream
	
	\item The stream itself has no integrity, so it requires an endorsement
	
	\item Though the stream is parametrised by the empty label, the stream \textit{itself} is confidential to the calling principal, requiring a declassification
\end{enumerate}

Through this methodology, it is possible to construct an output stream belonging to the calling principal which has no confidentiality. It is not, however, possible to construct a stream belonging to any other principal or with other confidentiality restrictions: even where JIF's typing system allows this, attempting to obtain the stream from the \mono{Runtime} instance will silently fail at runtime.

Hence, for practical purposes it is necessary to make full use of declassification as an `escape hatch' from the type system in order to actually allow the program to make useful output. This presents an issue as the type system can no longer track the flow of sensitive values after declassification, so earlier and more frequent use of the declassification construct presents more opportunity for accidental information leak bugs to arise.

\newpage

\subsection{Paragon Implementation}

\subsubsection{Data Types}

The Paragon version of Battleships is based on the same overall structure as the JIF implementation. The first notable difference, seen in the simple data type classes \mono{Coordinate} and \mono{Ship}, is that where JIF must rely on type parameters for polymorphism in the security-type system, Paragon has the special \mono{policyof(this)} annotation, allowing these classes to be fully polymorphic in policy without being explicitly parametrised.

As a side note, use of the \mono{policyof(this)} annotation is somewhat limited in practice -- it is available only for methods and \mono{final} fields. However, this suffices for the simple data types required by Battleships.

In addition, defining the overridden Java methods \mono{equals}, \mono{hashCode} and \mono{toString} for data types requires substantially less effort in Paragon than in JIF. JIF uses its own wrappers (\mono{JIFObject} and descendents) as it cannot easily represent polymorphic method parameters. In Paragon, the \mono{policyof(arg)} operator may be used to define a method's return policy in terms of a completely free argument policy.

The difference to annotation burden and readability can be easily seen in the method signatures of the JIF and Paragon equals methods respectively:

\textbf{JIF:}

\begin{minted}{Java}
public boolean{lbl;*lbl;L;o} equals(label lbl, IDComparable[lbl] o) {
\end{minted}

\textbf{Paragon:}

\begin{minted}{Java}
public ?(policyof(this) * policyof(o)) boolean equals(Object o) {
\end{minted}

\subsubsection{Confidentiality Policy}

As the focus of this comparison is on the use of information flow with respect to confidentiality, the integrity portion of the JIF program's design was not carried over into the Paragon implementation: hence, the Paragon implementation assumes the trustworthiness of the opponent's response to queries.

Though the Battleships game can quite easily be thought of in terms of interacting actors as under JIF's Decentralised Label Model, for the Paragon implementation modeling this was unnecessary. Paragon's `Objects as Actors' approach allows the \mono{Player} instance itself to be used as the actor in the security policy, without the need for an external, static principal. Thus, the secrecy of a player's board is determined solely by the following policy definition:

\begin{minted}{Java}
public final Player self = (Player)this;
public final policy boardPolicy = {
	  self :
	; Object o : allowBoardAccess(self)
};
\end{minted}

The first clause in the policy (\mono{self : }) states that only the current instance may view this information. The \mono{self} variable as opposed to the \mono{this} keyword is necessitated by limitations of the Paragon compiler.

The second clause of the policy allows for the declassification via the query system. \mono{allowBoardAccess(self)} is a unary paralock, taking a player as its argument. The policy clause states that any Object \mono{o} may access the board only if the \mono{allowBoardAccess} lock is open for the current player instance.

The core of this approach is that the lock remains closed at all times, \textit{except} briefly during the declassification process. While the lock is open, the board policy reduces to \mono{Object o:}, and at this point the required information is copied into a low confidentiality variable. Since this pattern is common (especially for implementations ported from JIF, where this form of declassification is the \textit{only} dynamic policy mechanism available), Paragon provides a special `declassification block' construct, as seen in the code below:

\begin{minted}{Java}
?boardPolicy boolean res = this.board.testPosition(query);
?bottom boolean declassified = false;

// Briefly open the lock to declassify the result
open allowBoardAccess(self) {
	declassified = res;
}

return declassified;
\end{minted}

Inside the scope of the braces, the lock is open, but it is closed everywhere else. Thus, information can \textit{only} leak inside of this block, and the compiler will prevent it from flowing down at any other point in the program.

While this is functionally equivalent to making use of a \mono{declassify} expression in JIF, the Paragon version is significantly more readable and makes the scope of the intentional information leak clearer.

\subsubsection{Boilerplate \& General Details}

Paragon does not require boilerplate to make use of standard output streams: there is a provided Paragon interface file for the Java \mono{PrintStream} class, which has a policy type parameter. It suffices to simply use \mono{System.out} as a \mono{PrintStream<\{Object x:\}>} (i.e. a stream with no confidentiality). Building output streams with more complex policies requires more work, but is not necessary for this case study.

As a result of the conciseness of the policy definition and the lack of boilerplate, the Paragon implementation of Battleships quite closely resembles a plain Java implementation, whereas the JIF implementation has large numbers of annotations on each function as well as additional steps and boilerplate code required to perform simple Java tasks. It is worth noting, however, that some of this need for extra annotation may be attributed to the inclusion of the integrity policy.

\subsection{Conclusion}

The Battleships examples can be represented under the policy models of both JIF and Paragon with relative ease. The interacting principals of the Decentralised Label Model can be adapted to the problem simply by mapping a principal to each player of the game. In the case of Paragon, these interacting actors may be implicitly modelled by the \mono{Player} instances themselves.

The declassification action required to allow for querying of the board aligns well with JIF's declassification construct: the query is an `escape hatch' for the confidentiality policy, and the \mono{declassify} statement is an `escape hatch' from the security type system. Paragon's system is more general than this, but the use of a briefly opened paralock to approximate declassification is common enough that Paragon includes syntactic sugar for it. Hence, in both languages this is quite easily expressed.

The largest difference between the two implementations lies in how Paragon's syntax reduces the need for complex boilerplate code. Using standard output, necessary for virtually all useful programs, requires non-trivial setup in JIF and none in Paragon. Similarly, Paragon provides the \mono{policyof()} operator to eliminate the need for security type parametrisation on simple datatypes -- and when Paragon does use security type generics, it extends the standard Java generics syntax where JIF uses entirely different syntax incompatible with Java generics.

Hence, while this example can be equivalently represented using both JIF and Paragon, Paragon's syntax produces a more concise and more readable program.