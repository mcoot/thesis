\section{Case Study 2: Conference Management System}

\textbf{Explanation}

Conference management system in which papers with a title and list of authors are submitted to the system. Allocations of papers to sessions are then performed, and ultimately the allocations of accepted papers are made publicly visible.

\textbf{Security properties:}

Before allocation release, authors (and indeed anyone bar the organiser) cannot be made aware of what session they are allocated to, or even \textit{if} their paper has been accepted and allocated to a session. (Modelled on real world leak in a system, as per LIFTy paper).

Before allocation release, the titles of submitted papers are visible but the author lists are not (for e.g. blind paper review).

\textbf{Key points:}

The `timed release' policy cannot be modelled in JIF (there is no way to model a time-variant policy).

Hence, for JIF, it has to be worked around with a declassification procedure -- meaning the compiler does not check whether it is valid for this procedure to be called (good example: switching the conditional on the session number function).

Paragon's Paralock construct very easily models timed release. The lock is used like a boolean expression and allows the compiler to easily detect a mistake like switching the conditional on the session number function.

Paragon can also model policy enforcement in two ways: by having a `default' value before information release (as with the session number function -- it may be called anywhere but will only produce meaningful information after release), or by statically checking the lock state (so attempting to call it from code not known to be after release will result in compile-time failure).

JIF can't model these two separate options -- only the former `default value' option can be encoded.

\textbf{Toy examples:}

Time-release policy

`Default value' versus compile-time checked

\newpage

\subsection{Overview}

The second case study is a simplified Conference Management system. Papers have a title and list of authors; these papers may then be submitted to the conference. Some submitted papers are then accepted and allocated to a conference session, in secret. Then at some point in time the final conference schedule is made available, and previously secret information about which papers were accepted becomes publicly available.

In addition, until the reveal the submitted papers' titles are publicly viewable, but their authors lists are secret (this could represent, for instance, a policy for blind review).

\subsection{Key Security Properties}

There are essentially two parts to the security policy required for this program, which both derive from essentially the same principle of `timed release':

\begin{enumerate}
	\item Until the timed release, whether a paper has been accepted or not may be known only to the conference organiser
	
	\item Until the timed release, the list of authors of a paper is not available to anyone except the conference organiser
\end{enumerate}

\subsubsection{Timed Release}

Timed release requires that confidentiality be time-variant -- equivalent pieces of code may be acceptable or unacceptable depending on \textit{when} they are called.

This is an inherently dynamic policy: it cannot be represented in a static lattice structure. Hence, its representation requires either some notion of declassification so that information can be allowed to `escape' from a lattice policy structure or some way of representing policy state directly. As will be discussed in the following sections, the former method is problematic as declassification bypasses the security-type system entirely, preventing the type system from properly enforcing the conditions of the timed release.

\subsubsection{Paper Acceptance}

The first aspect of the policy, that papers' acceptance status remain hidden until release, provides an illustration of how implicit information flows must be taken into consideration. It is relatively straightforward to prevent an explicit leak of this information (for example, from a method that returns whether or not the paper has been accepted), but in order to prevent \textit{all} flow, the security type system must also prevent any information whose value depends on the acceptance status from being leaked.

In the example application, this requirement is demonstrated through the ability for users to retrieve the session allocation for a given paper. Since any paper which has been allocated to a session must have been accepted, leaking the session allocation implies a leak of the acceptance status. This precise issue occurred in the EDAS conference management system \cite{agrawal2016edas_conf}, and is discussed in relation to an alternative approach to information flow by \citeauthor{polikarpova2016lifty} \cite{polikarpova2016lifty}.

Hence, the session allocation for a given paper must also be guarded at the same level of confidentiality as its acceptance status.

\subsubsection{Author Confidentiality}

The other aspect of the policy is more straightforward: until the timed release, papers' authors must be hidden from everybody except the conference organiser. This is novel when compared to the Battleships example in that the papers themselves --- their title, abstract and contents --- are not secret; so only \textit{some} information about the paper need be constrained by the timed release policy.

\subsection{Implementation Structure}

The JIF and Paragon implementations for the Conference Management case study were written to follow the same class structure. Each implementation attempts to provide essentially the same security policy mechanisms, but as discussed the degree to which the security typing can correctly encode the timed release mechanism varies between them.

The structure uses the following classes:

\begin{itemize}
	
	\item \mono{Author}: a simple immutable paper author with a name
	
	\begin{itemize}
		\item Polymorphic with respect to security policy
	\end{itemize}
	
	\item \mono{Organiser}: an immutable conference organiser with a name
	
	\begin{itemize}
		\item Polymorphic with respect to security policy
	\end{itemize}
	
	\item \mono{Paper}: an immutable paper with a title, abstract and list of authors
	
	\begin{itemize}
		\item Title and abstract may have a separate policy to the list of authors, though both are polymorphic
	\end{itemize}
	
	\item \mono{ConferenceSystem}: a conference representation with an organiser, a set of submitted papers, and a set of accepted / allocated papers
	
	\begin{itemize}
		\item Contains a list of submitted papers, as well as a map of accepted players to their allocated sessions, and provides the methods which perform the allocation in secret and perform the timed release
		
		\item Defines the security policy placed upon the acceptance status / session allocation, as well as that placed upon the author lists of submitted papers
	\end{itemize}
	
	\item \mono{Main}: a main class which creates and runs a conference system
	 
\end{itemize}

\subsection{JIF Implementation}

\subsection{Paragon Implementation}

\subsection{Conclusion}