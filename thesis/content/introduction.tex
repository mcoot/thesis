\chapter{Introduction}

The vast majority of computer applications and systems used today interact with confidential information, whether in the form of sensitive government or commercial documents, or in the form of customer data. Users rely on these applications to correctly enforce their intended confidentiality policies and thereby keep their data secure. In practice these applications frequently fail to correctly control confidentiality, often as a result of programmer error, or of edge cases in usage that were not considered in the program's design.

\newpage

\section{Simple, Common Confidentiality Violations}

A common class of edge case is the ``account enumeration vulnerability" \cite{accountenumeration}, which frequently appears on even the high-traffic, professionally developed websites. An account enumeration vulnerability arises from a usually subtle difference in the information presented to a user when they unsuccessfully attempt to log in. Often, failing to log in will present a message of the form ``The username you entered was incorrect", or ``The password you entered was incorrect". The intention is that the user should learn no information from failing to log in, except that the credentials they used were incorrect.

However, the messages presented above \textit{do} give the user more information: if they get the message ``The password you entered was incorrect", then they know that the username they entered is in fact a registered username, and if they receive the other message, they know it is not. An attacker can use this to try many possible usernames (i.e. to enumerate the valid accounts), making it significantly easier for them to gain unauthorised access to these accounts.

The effects of account enumeration vulnerabilities can be mitigated by limiting how frequently a single visitor may attempt to log into the application, but the vulnerability itself can be easily addressed by ensuring that the error message presented when a username is invalid is the same as that presented when a password is invalid. Yet, this vulnerability is consistently and repeatedly found on websites.

A more specific example of these information edge cases leading to confidentiality violations is a violation found in a well-known system for conference management \cite{agrawal2016edas_conf} \cite{polikarpova2016lifty}. In this system, papers are submitted to a conference and then either accepted, or rejected by a conference panel. Once accepted, papers are assigned to a particular conference session. Authors of a paper should be able to see which papers they have submitted, but they should not be able to learn whether or not their papers have been accepted before the conference is finalised.

The view which shows the papers an author has submitted contains a column for the session number of a paper, linked to the system's database. If the paper does not have a session yet, this column states ``\mono{(not assigned yet)}". But if the paper has been rejected then it no longer exists in the relevant part of the database, and so has no value for this column. An author viewing their papers can determine that any which state ``\mono{(not assigned yet)}" under the session allocation column have been accepted, and those which lack any information in that column have been rejected, and thereby the author can learn the status of the paper even when it is intended to be `pending'.

\section{Statically-Enforced Information Flow}

Confidentiality in systems is most often controlled through the implementation of \textit{access control mechanisms}, which perform runtime checks to determine whether data may be accessed by a particular user or at a particular point in the program's execution. These mechanisms are widely used, but they struggle to protect confidentiality in cases similar to the examples presented above. If the programmer did not think about the possibility that conference paper session allocation status could reveal the paper's acceptance status, why would they insert a runtime access check there?

The \textit{Information Flow} control mechanisms described in the following sections aim to provide confidentiality guarantees which for the most part do not require runtime checks; instead, a static analysis of the program is performed which determines the possible ways information can flow through it. This is checked against the stated policy, and only programs which \textit{provably} enforce their policy correctly will be accepted by the checker. In practice, this is implemented through the use of a `security typing' system, and a violation of the confidentiality policy corresponds to a type-checking compile failure.

Two extensions to the Java programming language which implement security typing, Java Information Flow (JIF) and Paragon, are examined and compared in the sections below, through the process of developing three `case study' applications which showcase particular security properties: an implementation of the game `Battleships', a conference management system, and a calendar scheduler application.

The evaluation aims to determine the advantages these languages have in terms of improved confidentiality guarantees over the languages popularly used for application development today, and also to examine the drawbacks these languages have, both in terms of confidentiality guarantees which they fail to provide and in terms of the additional burden they place upon a programmer attempting to develop an application using them.