\chapter{Conclusions \& Further Work} \label{chap_further_work}

\section{Information Flow}

Language-based information flow security has potential application in any program which enforces a confidentiality policy, and where programmer error has the potential to lead to accidental violation of a policy.

As this has been a topic of research for several decades, a number of security typing-based models have been developed. Each of these builds on those that came before and attempts to improve policy definition and enforcement. The model implemented by \citeauthor{denning1977certification} \cite{denning1977certification} allows for verification of programs under a standard MAC lattice. The Decentralized Label Model improves upon this by allowing policies with no central source of truth, and by framing policies in terms of interacting principals. The ParaLock model \cite{broberg2010paralocks} attempts to address the deficiencies of prior systems in defining dynamic policies in more sophisticated ways than straightforward declassification.

Using the tools these models provide it is possible to construct and enforce many useful security policies, from simpler systems which enforce a MAC lattice policy to those which require a dynamic `timed release' policy, and through the application of these techniques it is possible to mitigate the real-world risks of improperly enforced confidentiality policies which programs that interact with data in any meaningful way face.

\section{JIF \& Paragon}

The JIF and Paragon programming languages pair relatively mature implementations of security typed information flow control with the syntax and standard library of Java, one of the most widely used languages in existence. Both can provide powerful confidentiality guarantees through information flow checking which is primarily performed at compile time, avoiding the potential overhead of a runtime mechanism.

Both systems can be used to encode meaningful security policies. JIF's model allows for policies built around interacting principals and, crucially, does not require a central source of truth to determine what policy should be applied to information. Its declassification mechanism allows for a much larger class of programs, beyond those which are strictly non-interfering, to be written, with some level of control applied through the `authority' construct. 

Paragon's policy model is more general, and has a lower conceptual barrier as a result of basing itself upon the familiar constructs of predicate logic. Its use of Flow Locks and ParaLocks also allows for a wider range of dynamic flow policies than can be encoded using JIF's Decentralised Label Model.

\subsection{Case Studies}

The case studies presented in \autoref{chap_comparison} demonstrated both the application of the two languages and their limitations.

\subsubsection{Battleships}

The Battleships case study presented in \ref{sect_cs_battleships} requires modelling a `game board' containing information secret to one of the game's players. This information must not leak to the other player, except through a specific querying mechanism as part of the game's rules. This policy is easily encoded in JIF by representing each player as a principal, and using a declassifying method to handle queries. Paragon similarly encodes this; it does not require an explicit principal for each players, because Paragon uses `objects as actors' -- the player instance itself \textit{is} the principal. Paragon performs declassification using a simple flow lock, cleanly emulating the same declassification strategy as the JIF implementation.

\subsubsection{Conference Management}

The Conference Management case study presented in \ref{sect_cs_conf} models a `timed release' policy, where the confidentiality status of session allocations for a paper submitted to a conference changes during the system's operation. Until the conference gives notification, the status is secret such that authors cannot learn whether their paper was accepted; after the notification the status is public information.

JIF's Decentralised Label Model cannot naturally encode this policy; timed release cannot be expressed in terms of interacting principals. As a result, the JIF implementation of this case study relies on using declassification as an `escape hatch' from the security typing system. Hence, developing this application in JIF is feasible, but doing so requires giving up some of the confidentiality guarantees that security typing provides.

Paragon, by contrast, encodes this policy with a single flow lock, which is opened when the timed release condition is met. Hence, the Paragon implementation of this application is able to provide more stringent confidentiality guarantees than the JIF version.

\subsubsection{Calendar Scheduler}

The Calendar Scheduler case study presented in \ref{sect_cs_csched} models a policy where the confidentiality of a meeting's details depend on the runtime list of attendees. Here, neither JIF nor Paragon could usefully encode the desired confidentiality policy.

JIF's policy model fundamentally cannot express the concept of confidentiality dependent upon a dynamic \textit{set} of users. A class may be parametrised by a fixed number of principals, and runtime principals and labels may be used under some circumstances, but the type system is unable to actually encode or enforced the policy required to provide meaningful verification of the confidentiality properties of this application.

Paragon comes closer to encoding the policy; its policy definition language allows for universal quantification over objects, and a parametrised lock may be used to denote attendance status. However, the compiler's type checker is then not able to properly enforce this policy and as a result, Paragon cannot provide the desired security guarantees.

\section{Further Work}

The analysis of the JIF programming language presented in this thesis focused on development using recent versions of JIF and the standard JIF runtime. The JPmail project \cite{jpmailpage} at Pennsylvania State University developed a set of policy tools for JIF which were not thoroughly examined for this comparison. These tools include an Eclipse plugin, JIFclipse, along with tools for supporting a policy store with dynamic look-up of principals. However, these tools were designed for a significantly outdated version of JIF and require the use of a modified JIF runtime, and the project is no longer under active development.

In the literature, there are also other approaches to alleviating some of the limitations of JIF. One such approach is that taken by the RX language \cite{swamy06rx}, which builds upon the Decentralised Label Model but attempts to introduce support for dynamic policy through `policy updates', using both static analysis and runtime checks to reason about the information flows through a program. Though the principles of the RX language have been developed through two papers \cite{hicks05rx} \cite{swamy06rx}, it has not been implemented in practice.

Another system, Java Object-sensitive ANAlysis (JOANA) \cite{graf2013joana}, makes use of a separate static analyser and annotations in Java code, rather than implementing information flow checks by extending the type system. It analyses Java bytecode against traditional lattice-based policies by generating a `dependency graph' of flows through the system. This approach is is built on substantially different theoretical foundations to those of JIF and Paragon, and it has potential upsides in requiring lower programmer burden (requiring no syntax beyond that of standard Java, excepting additional method annotations). As a result, it would make a potentially useful comparison with more traditional security typed languages if performed in depth.

In a 2016 paper \citeauthor{kozyri2016jrif} \cite{kozyri2016jrif} introduce an approach to information flow with dynamic policy which aims to address limitations in JIF, based on `Reactive Information Flow (RIF) automata'. Whereas a DLM label as used in JIF encodes a set of principals who may read the data, a RIF label encodes a finite state automaton, with each state associated with a set of principals who may read the data \textit{when the automaton is in that state}. Dynamic policies are represented through the use of transitions in these automata.

The paper \cite{kozyri2016jrif} introduces JRIF, a language based upon JIF which implements RIF automaton-based security typing, and presents example implementations of Battleships, and a calendar application (which is conceptually similar to the Calendar Scheduler case study presented in \ref{sect_cs_csched}). The JRIF language potentially presents a way to resolve many of the issues encountered with JIF's label model through the development of the three case studies in this thesis; however, the JRIF label syntax is substantially more complex even than JIF and so it may not be a panacaea to the programmer burden issues encountered with JIF.

A radically different approach to security typed information flow is that implemented by \citeauthor{polikarpova2016lifty} \cite{polikarpova2016lifty} with `Liquid Information Flow Types' (LIFTy). LIFTy is an extension of the Haskell programming language that combines information flow with program synthesis techniques: rather than \textit{verifying} a program as respecting an information flow policy, the compiler synthesises runtime checks to enforce them. Hence, the language is `policy agnostic': the user does not need to annotate their code at all, but merely specify the policy separately. The compiler will then insert runtime checks that ensure the policy may never be violated.

These newer advances in the information flow literature (and the implementations of them) are for the most part quite recent, and lack the broad supporting literature or the maturity of implementation provided by JIF or Paragon. Nonetheless these systems are designed with the limitations of earlier security typing-based languages in mind, and may well play a part in addressing the issues that make JIF and Paragon impractical for application development in most contexts.

\section{Project Reflection}

%The development of the overall comparison changed significantly in scope and in direction over the project.

\subsection{Focus of Evaluation}

Initially, several approaches to the idea of evaluating information flow security were considered. In particular, one approach considered was to compare the use of information flow-based security with specific, existing mechanisms for access control. This approach would have the advantage of providing a direct comparison with systems (such as Java's security model) which are both well understood and which are actively used within industry. However, there are significant differences in the \textit{intent} of these different models, and hence in the kinds of problems they attempt to solve.

Java's security model focuses primarily on controlling the potential actions of \textit{code}, rather than the confidentiality of \textit{data} per se. These are related, but they have distinct purposes. For instance, information flow mechanisms do not attempt to resolve concerns about the trustworthiness of code which is loaded dynamically at runtime, potentially sourced from the internet, but this is fundamental to Java's security model. And similarly, the actual information revealed by a program's execution through both variables in memory and implicit flows is not a problem considered by the Java security model.

A second approach was to focus on the ways in which information flow mechanisms could impact on or mitigate the effects of security vulnerabilities within Java itself and commonly used Java libraries. This approach would have involved examining known vulnerabilities listed in the Common Vulnerabilities and Exposures (CVE) system, and relating them to the guarantees provided by an information flow system.

This approach was also problematic, as many vulnerabilities do not directly relate to the flow of information, and those that do primarily fall outside the realm of what can be addressed by existing information flow implementations. For instance, they often rely on reflection, which security typed languages like JIF and Paragon do not implement.

Ultimately, the approach shifted to considering information flow mechanisms on their own terms, with a couple of guiding questions: first, ``Are they useful?", and second, ``Which mechanisms are most useful?". The first question is not easily answered directly; there are clear applications for information flow which derive by its nature, but their usefulness more broadly is inherently tied to the specific implementation, and hence to the second question, which became the focus of the project.

The key decision which flows from this approach is the question of which mechanisms to consider. Information flow encompasses a number of related fields, but even within the scope of \textit{language-based} information flow security, a number of mechanisms have been proposed and implemented. It was clear that JIF should form a part of any comparison -- it is by far the most mature implementation of language-based information flow, and is used as a baseline in many other parts of the literature. The decision was made to stick to `mostly static' implementations of information flow, as runtime-enforced information flow tends to be focused on web technologies (for instance, the FlowFox web browser \cite{degroef2012flowfox}) rather than on languages like Java.

Paragon was then a natural language to compare against, as it explicitly describes itself as a `third generation' security typed language, representing a generational leap from `second generation' languages like JIF.

\subsection{Comparison Methodology}

The driving reason behind the decision to compare only two security typed languages, JIF and Paragon, was that in order to meaningfully evaluate how these languages can be used in practice it is necessary to spend time developing non-trivial programs with them. Attempting to do this thoroughly with more than two languages would have been impractical under the project's time constraints, though this does result in some promising avenues (such as the RJIF \cite{kozyri2016jrif} and JOANA \cite{graf2013joana} systems) not being explored.

Early in the project, it was anticipated that the comparison would centre around the development of a single large application, and the from this development analysis of both the security properties and programmer burden of each language could be drawn. One such candidate for this was to build a working email client -- similar to that developed by the JPmail project \cite{jpmailpage}. However, it became apparent quite quickly that developing an application of this scale would require a substantial level of skill in developing with security typed languages, and that neither JIF nor Paragon was mature enough for this implementation to be practical -- note that the JPmail project had to implement new tools and modify the JIF runtime in order to complete their implementation.

The final approach, to instead build three more modestly sized case studies, was substantially more practical given the constraints of the project and also allowed for a more meaningful comparison between different kinds of security policies that might be desired by application developers.

The Battleships case study was chosen first, as the JIF implementation is bundled as an example with the JIF compiler, making it a natural program to use for comparison. The Conference Management System case study was selected based on its prevalence as an example in the literature \cite{agrawal2016edas_conf} \cite{polikarpova2016lifty}. Finally, the Calendar Scheduler was selected as it represents a commonly used program design which necessitates a more complex form of dynamic policy.

\subsection{Conclusion}

Overall, though the general project direction took a significant amount of time to stabilise, the resulting comparison based on case studies produced clearer and more insightful comparisons than an attempt to build a single larger application would have. However, the limitation in scope caused by restricting the comparison to be only between JIF and Paragon produces some limits on the usefulness of the conclusions drawn, since more recent and less mature attempts to address the issues exposed by the case studies are not thoroughly considered.

In addition, lack of familiarity with the languages and concepts made the development of the case studies relatively slow and error-prone. JIF and Paragon both have a high conceptual burden, and as documented in \ref{sect_practicalities} merely setting up these languages for development is a non-trivial task. Skills and familiarity with the languages were developed incrementally over the lifespan of the project, and so the case studies could have been improved, made clearer and designed with more sophisticated confidentiality properties if developed after a more in-depth knowledge of the languages and the theory was obtained.

Nonetheless, these languages are widely used within the literature of the field of information flow, and crucially, there is relatively little literature about the application of these languages to developing real-world programs. Hence, there is value in comparing and evaluating these languages on those terms, and case studies provide a methodology for performing this comparison in a way which both provides a wide range of confidentiality problems and relates to practical application development.

