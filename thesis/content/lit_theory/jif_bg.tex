\section{JIF}

\subsection{Relation to Plain Java}

JFlow, a Java language extension implementing the Decentralised Label Model (DLM) via type checking was first proposed in \citetitle{myers1999jflow} \cite{myers1999jflow}. It included DLM-based type labels with support for polymorphism and type inference, while integrating with Java's Object-Oriented programming model -- including language features like inheritance and exceptions which introduce potential difficulties for information flow checking.

JIF implements a superset of the JFlow functionality, having added a number of additional features over its development \cite{jifwebsite}. Since JIF's 1.0.0 version released in 2003 its design is based on Java 1.4, and so it lacks support for a number of more modern Java features.

In general, working with JIF has a number of differences when compared to Java. Some notable ones include:

\begin{itemize}
	\item JIF places DLM labels on fields, method signatures and potentially local variables
	
	\item The special \mono{declassify} statement (and its integrity equivalent, \mono{endorse}) may be used within the code
	
	\item \textit{All} exceptions are checked in JIF (including runtime exceptions such as \mono{NullPointerException})
	
	\item When running a JIF program, the JIF runtime must be on the classpath
	
	\item Interacting with (non-JIF) Java classes requires writing a JIF interface indicating labels on the class's methods
	
	\item Console I/O using stdin / stdout is performed through JIF's runtime
	
	\item JIF lacks support for the angle bracket generics syntax introduced in more recent versions of Java
\end{itemize}

\subsection{Policy Model}

The \textit{decentralised} nature of the DLM is its primary innovation over previous models. Mandatory Access Control requires that each unit of information -- such as a variable, or method return value in Java -- has a label universally agreed upon by all users of the system. The DLM instead models `principals' in the system (which may represent individual users, groups of users or roles), which do not need to trust each other.

The model uses an \textit{acts-for} relation between principals -- if principal $ p $ acts for principal $ q $, then any action taken by $ p $ is assumed to be authorised by $ q $ \cite{myers2000dlm}. Principals may also be composed conjunctively as $ p \& q $ or disjunctively as $ p,q $. The composite principal $ p\&q $ acts for both $ p $ and $ q $, while $ p $ and $ q $ each act for composite principal $ p,q $.

In practice, each principal in JIF is represented by a Java class implementing the \mono{jif.lang.Principal} interface, allowing the programmer to define new principals and specify which other principals they act for.

Confidentiality labels in JIF consist of terms of the form \jiflabel{o->r1,r2}, where \mono{o} is the principal expressing ownership over the policy. When evaluated, these labels produce a \textit{reader set} of principals that the owner believes may read the data -- in this case, \mono{r1} and \mono{r2} and itself, \mono{o}. The reader sets produced form a lattice, and so labels may be combined via the \textit{join} (least upper bound -- ``\mono{;}") and \textit{meet} (greatest lower bound -- ``\mono{meet}") operators, which produce the intersection and union of the constituent labels' reader sets, respectively.

Most commonly, the join operator is used to allow multiple principals to provide differing policies on data. This produces a label of the form:

\mint{java}{{o1->r1,r2; o2->r2,o1,r3}}

The join of each policy produces a reader set which is the \textit{intersection} of both -- so in this case, only \mono{r2} and \mono{o1} (or principals which act for either) may read the information. Note that neither \mono{o1} nor \mono{o2} is the `true' owner of the data: each is simply expressing what they believe the policy to be.

In addition to producing new labels via the join and meet operators, there exists a top principal (written \code{*}), which is a principal for which no other principals act, and a bottom principal (written \code{_}), for which all other principals act. A reader set consisting of the top principal allows \textit{nobody} to read the information, and a reader set consisting of the bottom principal allows \textit{anyone} to read the information.

In practice, a label term of the form \jiflabel{Alice->*} indicates that Alice believes that only she should be in the reader set; a label term \jiflabel{Alice->_} indicates that Alice believes that everyone should be in the reader set.

The top label \jiflabel{*->*} represents the highest possible confidentiality (essentially allowing no principal to read it), while the bottom label \jiflabel{_->_} represents the lowest confidentiality; an empty label \jiflabel{} is equivalent to the bottom label.

\subsection{Implicit Flows}

Since information may leak to a lower confidentiality implicitly via flow of control, JIF has a program counter or pc label which represents the confidentiality status of the current block of code. For instance, within a branch of an \mono{if} statement with a conditional dependent on a high confidentiality value, the pc label will have high confidentiality.

\subsection{Label Annotations}

The below shows JIF labels on a field (as in \jiflabel{P->_} above), indicating that field's confidentiality.

\begin{minted}{Java}
public String{P->_} myField;
public int{A->*} method{B->*}(String{C->*} param) : {D->*}
\end{minted}

JIF also places labels on method signatures, for their return type (as in \jiflabel{A->*} above) and formal parameter types (as in \jiflabel{C->*}); a formal parameter's label represents an \textit{upper bound} on the allowed label of its actual argument. The begin label (as in \jiflabel{B->*}) states the \textit{required} pc label for calling code, while the end label (as in \jiflabel{D->*}) indicates what the pc label will be once the code completes.

Local variables may be annotated with labels, but JIF uses type inference to attempt to determine the correct label for local variables automatically. If labels are omitted on fields or method signatures, a default label is applied: for fields and method end labels this is the bottom or empty label \jiflabel{}, while for formal method arguments and method begin labels it is the top label \jiflabel{*->*}. For return labels, the default label is the join of all parameter labels and the end label.

\subsection{Declassification}

It is allowable for the label of a value or variable in JIF to change \textit{if} that re-labelling results in a label which is strictly higher confidentiality with respect to the DLM lattice. Hence, restricting readers or joining an additional policy to a label is always acceptable, as is adding a new reader \mono{r} if there already exists a reader in the policy that \mono{r} acts for. For instance, the following is valid:

\begin{minted}{java}
int{Alice->Bob,Charles} x = 1;
int{Alice->Bob} y = x;
int{Alice->Bob,Charles; Bob->*} z = x;
\end{minted}

However, most useful programs must leak some information to lower confidentiality in order to perform useful action. JIF provides the declassification construct in order to enable this, while still providing some controls: a user must provide \textit{authority} to a method in order to declassify data within it, and they may only declassify policy terms which they own.

\begin{minted}{java}
public static void declassifyInt{Alice->*}() where authority(Alice) {
	int{Alice->*} high;
	int low{Alice->_} low;
	
	low = declassify(high, {Alice->*} to {Alice->_});
}
\end{minted}

In the above, the method operates with Alice's authority, and is therefore able to allow the information stored in \mono{high} (which only Alice can read) to move to \mono{low} (which anyone can read).

\newpage

\subsection{Sample Class: Diary} \label{jif_bg_sample_diary}

The class in \autoref{lst_diary_jif} implements the policy scenario from \ref{theory_dac_limitations}. Alice has ownership over the diary, and allows Betty to read it. The program will not let Betty propagate access to Mal: a program which did would not compile.

JIF's type system ensures that explicit and implict flows are valid, as in the \mono{readDiary} and \mono{isTheSecret5} methods respectively. It includes the \mono{declassifyDiary} method; only by an explicit call to this method can Mal gain access.

\begin{listing}[!ht]
	\inputminted{java}{content/code_sections/jif_para_bg/Diary.jif}
	\caption{JIF Diary Implementation}
	\label{lst_diary_jif}
\end{listing}

\clearpage

\subsubsection{Compilation Failure in JIF}

Consider the modified (and abridged) version of the previous class in \autoref{lst_diary_jif_error}:

\begin{listing}[!ht]
	\inputminted{java}{content/code_sections/jif_para_bg/DiaryError.jif}
	\caption{Erroneous JIF Diary Implementation}
	\label{lst_diary_jif_error}
\end{listing}

The only change from the original \mono{Diary} class is the return label of the \mono{readDiary} method, allowing Mal to read the result of the method call and violate the information flow policy. This causes a compiler error, as per \autoref{lst_diary_jif_error_message}.

\begin{listing}[!ht]
	\inputminted[fontsize=\footnotesize]{text}{content/code_sections/jif_para_bg/DiaryError_message_jif.txt}
	\caption{Erroneous JIF Diary Implementation: Compiler Message}
	\label{lst_diary_jif_error_message}
\end{listing}

\clearpage

\subsection{Integrity Controls}

From version 3.0.0, JIF includes a set of integrity controls in addition to the confidentiality controls discussed here. These controls are beyond the scope of this thesis and so will not be examined in-depth. In brief, the integrity lattice is much like the confidentiality lattice, except rather than being ordered by how high the confidentiality is, integrity labels are ordered by how \textit{trusted} they are, and information cannot flow from untrusted to trusted context.

\subsection{Security Type Polymorphism}

Since all data must be labelled (or have a label determined implicitly via type inference), writing implementations of standard data types presents a challenge -- it is desirable to avoid having to rewrite the \mono{ArrayList} class for every possible security label the elements of the list may have. To deal with this, JIF introduces security type generics. Since JIF's syntax was designed prior to Java's adoption of generics, square brackets are used to denote type parameters (as opposed to angle brackets).

JIF provides two different kinds of generics: \textit{label} generics and \textit{principal} generics.

A class with a label generic, of the form \mono{TestClass[label L]} may be parametrised by any label. The label type parameter may be used anywhere within the class (e.g. as labels on fields, method return values or begin/end labels). Using this, a data type like \mono{ArrayList[Label L]} may be written such that it can be instantiated with any confidentiality label on the elements.

The second kind of generic is the principal generic, of the form \mono{TestClass[principal P]}. The principal type parameter may be used in labels in the class. This allows for classes which have explicit policies, but which are generic in being `owned' by any principal.

\subsection{Runtime Features}

Though JIF enforces its information flow policy primarily statically, it does provide runtime features. Labels and principals may be used as first class values throughout code, and so must exist at runtime. First-class (or \textit{dynamic}) labels may be used in specifying other labels and as type parameters, which are still evaluated statically through dependent types. Note that dynamic labels do not, for the most part, allow for the kinds of `dynamic policy' that the policy model of Paragon revolves around.

A runtime label value may be constructed with an expression of the form:

\mint{java}{final label lbl = new label {Alice->Bob};}

JIF provides runtime testing of both labels and principals. Labels may be compared in a conditional statement of the following form, which will execute the block if \code{lbl} is of lower or equal confidentiality than \code{new label {Alice->*}}:

\begin{minted}{java}
if (lbl <= new label {Alice->*}) {
	...
}
\end{minted}

Code within the block can then safely assume that \code{lbl} is of lower or equal confidentiality than \code{new label {Alice->*}}, and will compile under that assumption.

Similarly, principals may be compared using a conditional statement of the following form, executing the block if \code{Alice} acts for \code{Bob}:

\begin{minted}{java}
if (Alice actsfor Bob) {
	...
}
\end{minted}

Within the block, it is safe to assume that \code{Alice} does in fact act for \code{Bob}.