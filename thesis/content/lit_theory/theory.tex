\chapter{Background \& Theory} \label{chap_theory}

\section{Security Context}

Information security in practice revolves around three key goals colloquially known as the `CIA Triad' \cite{krutz2010cloudsec}:

\begin{itemize}
	\item Ensuring information is appropriately \textit{Confidential}

	\begin{itemize}
		\item Example: a secret password being leaked is a confidentiality violation
	\end{itemize}
	
	\item Ensuring information has \textit{Integrity}
	
	\begin{itemize}
		\item Example: a database being accessed and records falsified is an integrity violation
	\end{itemize}
	
	\item Ensuring information is \textit{Available}
	
	\begin{itemize}
		\item Example: a Denial of Service (DoS) attack causing a website to crash is an availability violation
	\end{itemize}

\end{itemize}

Most language-based security features focus on the confidentiality and integrity of information; that is, ensuring that secret information remains secret, and that information which needs to be trusted is, in fact, trustworthy.

The most common mechanism used in computer systems to ensure confidentiality and integrity is \textit{access control}: associating some `permissions' with users of a system and restricting the actions they are able to perform based on these permissions.

\section{The Java Security Model}

The Java programming language was designed in the context of emerging internet technologies, with the explict use case of users downloading and running `applets' from web pages. As such, security was a core consideration of Java's design -- the original design goals document \cite{javadesignprinciples} specifies that ``applications written in the Java programming language are secure from intrusion by unauthorized code."

Java is type safe and memory safe, which reduces the possibility of programmer error leading to exploitable flaws in an application. Java's application sandbox takes a three-pronged approach to security in the context of applets which may be downloaded from a remote server, with the Bytecode Verifier, the Class Loader, and the Security Manager \cite{mcgraw1999securingjava}.

The Bytecode Verifier and the Class Loader concern the loading of new classes into a running Java Virtual Machine (JVM). The Bytecode Verifier analyses the class to ensure it maintains the class format, does not perform illegal casts, and that it obeys Java's typing rules more generally (e.g. that private methods may not be accessed outside of a class, final methods may not be overridden) \cite{lindholm2014java}. The Class Loader performs the actual loading of a class into the JVM; the `primordial' Class Loader \cite{mcgraw1999securingjava} loads the Java API classes, and other classes may be loaded by a user-specified \mono{ClassLoader} instance.

\subsection{The Security Manager \& Stack Inspection}

The final element of the sandbox model is the \mono{SecurityManager} class. When enabled, an instance of \mono{SecurityManager} runs in the JVM, and performs runtime access control: when a potentially sensitive action is undertaken, the Security Manager checks the permissions of the calling class based on the system policy (usually specified by an external policy file), and throws an exception if the permission check fails \cite{gosling2014java}.

Permission checks are performed using stack inspection \cite{gong2003javasecurity}: every frame on the call stack below the sensitive operation is examined, and if \textit{any} frame does not have the required permissions, the check fails. Java provides the \mono{doPrivileged} construct to bypass full stack inspection \cite{gong2003javasecurity} where this is desired functionality.

\section{Mandatory Access Control \& The Lattice Model}

Access control as implemented in many programs and operating systems (for example: the Unix permissions system) depend on Access Control Lists specifying what items individual users or groups of users may access. These models are known as \textit{Discretionary} Access Control.

By contrast, \textit{Mandatory} Access Control (MAC) is most commonly associated with the military and other high-security organisations. In a MAC system, all data has a \textit{classification}, and users operate with a \textit{clearance}. 

In the simplest case, the set of classifications is just an ordered list -- for instance, `Unclassified' $ < $ `Classified' $ < $ `Secret' $ < $ `Top Secret'. A user with `Secret' clearance, then, cannot access or modify `Top Secret' documents.


BELL LAPADULA LATTICE

TWO PROPERTIES

HIGH WATER MARK

\section{Information Flow Security}

FLOW RATHER THAN ACCESS

DENNING MODEL

\section{Formal Non-interference}

	\subsection{Implicit Flows}
	
	\subsection{Undecidability in the General Case}

	\subsection{Covert Channels \& Attacker Model}

\section{Declassification}

\section{Enforcement}

\subsection{Statically Enforced Models}

DENNING

DLM

PARAGON

\subsection{Runtime Enforced Models}




%\section{The Decentralised Label Model}
%
%\section{Logic-Based Policies \& The Paralock Model}
%
%\section{Other Models}


%** MOVED FROM COMPARISON... **
%
%The `JFlow' language, through both the paper \cite{myers1999jflow} that proposed it and its subsequent implementation as Java Information Flow (JIF), has become the dominant language and the dominant paradigm in security typing and static information flow control languages more generally.
%
%As articulated by \citeauthor{broberg2013paragon} in \citetitle{broberg2013paragon} \cite{broberg2013paragon}, JIF may be regarded as a `second-generation Information Flow language'. Its Decentralised Label Model allowed for more flexible and more useful policies to be expressed than the straight Mandatory Access Control lattice that had been at the centre of most prior work, and its implementation as an extension of the popular Java language made it comparatively practical to work with.
%
%Under this taxonomy, Paragon is then a `third-generation Information Flow language'. It discards lattice-based policy definition entirely in favour of policies which may defined through logical expressions. This abstraction, combined with the Paralock construct which allows for policies which model relations and which vary over the lifetime of a running program, broadens the scope of what security requirements can be expressed.
%
%The Paralock construct which Paragon introduces allow for policies which vary over the lifetime of a running program, something which cannot be represented effectively under JIF's policy model. In addition, Paralocks can model relations between actors of arity zero, one or two, and through the use of binary relations, the Decentralised Label Model can in fact be written within Paragon's policy language. That is, JIF's policy mechanism can be encoded using Paragon's.