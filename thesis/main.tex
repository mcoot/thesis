\documentclass[12pt,openany,a4paper]{book}

%%%%%%%%%%%%%%%%%%%%%%%%%%%%%%%%%%%%%%%%%%
%% Packages
%%%%%%%%%%%%%%%%%%%%%%%%%%%%%%%%%%%%%%%%%%
\usepackage[style=ieee]{biblatex}

\usepackage[english]{babel}
\usepackage{geometry}
\usepackage[hidelinks]{hyperref}
\usepackage{nameref}
\usepackage[parfill]{parskip}
\usepackage{listings}


\usepackage{algorithmicx}
\usepackage{algpseudocode}
\usepackage[titles]{tocloft}
\usepackage{newfloat}
\newlistof{listing}{lol}{List of Listings}
\usepackage{caption}
\usepackage{minted}
\setminted[newfloat]{tabsize=4, linenos}

\newenvironment{codelst}{\captionsetup{type=listing}}{}
\SetupFloatingEnvironment{listing}{%
	name={Listing},
	fileext=lol}

\renewcommand{\cftfigpresnum}{Figure~}
\setlength{\cftfigindent}{0pt}
\setlength{\cftfignumwidth}{2cm}

\renewcommand{\cftlistingpresnum}{Listing~}
\setlength{\cftlistingnumwidth}{2cm}

\usepackage{graphics}

\usepackage{titling}
\usepackage{titlesec}

\usepackage{subcaption}

\usepackage{tikz}
\usetikzlibrary{automata}
\usetikzlibrary{arrows}
\usetikzlibrary{arrows.meta}
\usetikzlibrary{decorations, decorations.pathreplacing, shapes}
\usetikzlibrary{calc}

\tikzset{
	%	vertex/.style={circle,draw,minimum size=1.5em},
	%	main node/.style={circle,draw,font=\sffamily\Large\bfseries},
	%	main node/.style={circle,thick,draw},
	edge/.style={->,> = latex'}
}

\tikzset{
	dot/.style={fill, draw, circle, minimum width=10pt, inner sep=0pt, scale=0.3}
}


\addto\extrasenglish{%
	\renewcommand{\chapterautorefname}{Chapter}%
}

%\usepackage{showframe}

%%%%%%%%%%%%%%%%%%%%%%%%%%%%%%%%%%%%%%%%%%
%% Settings
%%%%%%%%%%%%%%%%%%%%%%%%%%%%%%%%%%%%%%%%%%

\titlespacing*{\subsubsection}
{0pt}{\baselineskip}{0.5\baselineskip}

%%%%%%%%%%%%%%%%%%%%%%%%%%%%%%%%%%%%%%%%%%
%% Macros
%%%%%%%%%%%%%%%%%%%%%%%%%%%%%%%%%%%%%%%%%%

\newcommand{\mono}[1]{\texttt{#1}}

\newcommand{\monotuple}[1]{$ \left\langle \right. $\mono{#1}$ \left. \right\rangle $}

% Mint code highlighting
\newcommand{\code}[1]{\mintinline{java}{#1}}

% JIF
\newcommand{\jiflabel}[1]{\code{{#1}}}

% Paragon
\newcommand{\paralabel}[2]{\jiflabel{#1: #2}}
%\newcommand{\generic}[1]{$ < $#1$ > $}

%%%%%%%%%%%%%%%%%%%%%%%%%%%%%%%%%%%%%%%%%%
%% Document
%%%%%%%%%%%%%%%%%%%%%%%%%%%%%%%%%%%%%%%%%%

\begin{document}

\begin{frame}
	\titlepage
\end{frame}

\begin{frame}{Outline}
	\tableofcontents
\end{frame}

\section{Introduction}

\begin{frame}{What even}
	\begin{itemize}
		\item Java is used
	\end{itemize}
\end{frame}
\section{Background}

\begin{frame}{Information Flow Basics}
	Information Flow considers \textit{confidentiality states}. Flow \textit{policies} formalise how information may move between states.
	
	Simple model: logical predicates.
	
	A more complex model: the Bell-LaPadula Lattice model (as used by the US military).
	
	\begin{figure}
		\includegraphics[scale=0.45]{content/images/lattice_examples.png}
		\caption{Example Lattice Model states \cite{ifbackground:sandhu}}
	\end{figure}
	
\end{frame}

\begin{frame}{Noninterference}
	
	A program which leaks no high confidentiality information is `noninterfering', but proving that a program is noninterfering is usually not practical.
	
%	\begin{block}{Noninterference}
%		A program is noninterfering if any two different executions which differ only in their high confidentiality inputs are indistinguishable to an attacker. \cite{ifbackground:goguen}
%	\end{block}
	
	\begin{itemize}
		\item Proving noninterference (in the general case) is undecidable
			\begin{itemize}
				\item The halting problem can be reduced to it
			\end{itemize}
		\item Many useful programs are inherently interfering
			\begin{itemize}
				\item A password checker's output clearly depends on the password
			\end{itemize}
	\end{itemize}
	
	Most real systems allow for \textit{selective declassification}.
\end{frame}

%\begin{frame}{Noninterference - Is It Practical?}
%	Problems with proving noninterference:
%	
%	\begin{enumerate}
%		\item Proving non-interference (in the general case) is undecidable
%			\begin{itemize}
%				\item The halting problem can be reduced to it -- consider: \newline \texttt{\textbf{if} S() halts \textbf{then} h := 1 \textbf{else} h := 0} \cite{ifbackground:denninghalting}
%				\item \textit{Most real systems prove a simplified case using type checking}
%			\end{itemize}
%		\item Many useful programs are inherently interfering
%			\begin{itemize}
%				\item A password checker's output clearly depends on the password
%				\item \textit{Most real systems allow for `selective declassification'}
%			\end{itemize}
%		\item Noninterference does not model covert channels
%			\begin{itemize}
%				\item \textit{Some systems secure against specific potential covert channels}
%				\item There is no way to prove that no covert channels exist
%			\end{itemize}
%	\end{enumerate}
%\end{frame}

\begin{frame}{Enforcement: Static or Dynamic?}
	Flow controls be enforced at compile-time or at run-time.
	
	\textbf{Dynamic} information flow controls:
	\begin{itemize}
		\item Flexible: can represent policies which change dynamically
		\item Incur some run-time overhead
		\item Can only track the code path which is executed
	\end{itemize}
	
	\textbf{Static} information flow controls:.
	\begin{itemize}
		\item Have little or no run-time overhead
		\item Can easily track \textit{all} possible code paths
	\end{itemize}
	
	Most solutions use static or `mostly static' approaches.
\end{frame}

%\begin{frame}{Information Flow and Integrity}
%	Information flow is usually talked about with respect to confidentiality, but it can also be applied to integrity.
%	
%	For \textbf{Confidentiality}:\newline Track the flow of high confidentiality (`secret') outputs.
%	
%	For \textbf{Integrity}:\newline Track the flow of low integrity (`tainted') inputs.
%\end{frame}
\section{Thesis Progress \& Plan}

\begin{frame}{Work So Far}
	Thus far, most of the work on my thesis has been split between three tasks:
	
	\begin{enumerate}
		\item Familiarising with one particular implementation: JIF (Java Information Flow)
		\item Researching information flow models and implementations
		\item Researching the applicability of information flow to real security problems
	\end{enumerate}
\end{frame}

\begin{frame}{JIF}
	Java Information Flow is a language extension to Java with `mostly static' information flow features through `security types', which are policies as per the Decentralised Label Model \cite{work:myersdlm}.
	
	Every variable has a security label attached to it which encodes how its information may flow between principals. For instance:
	
	\texttt{int\{Alice->Bob\} x;}
	
	indicates that principal Alice owns the information, and allows that information to flow to Bob or a principal Bob has delegated authority to.
\end{frame}

\begin{frame}{JIF: A Minimal `Hello World'}
	\begin{figure}
		\includegraphics[width=\linewidth]{content/images/jif_helloworld.png}
	\end{figure}
\end{frame}

\begin{frame}{JIF: Static Checking Errors}
	\begin{figure}
		\includegraphics[scale=0.5]{content/images/jif_helloworld_error.png}
		\caption{A JIF compiler error}
	\end{figure}
\end{frame}

\begin{frame}{JIF: Practicality \& Programmer Burden}
	
	JIF's issues:
	
	\begin{itemize}
		\item Inherently complex policies
		\item Programmer burden compounds as complexity increases
		\item Sparse documentation, no debugger and misleading compiler errors
		\item Interoperability with Java
	\end{itemize}
	
	\note{
		Thus far in my research, JIF's largest problem has proven to be the complexity of its policies and the programmer burden that induces.
		
		JIF inherently places a large cognitive burden on the programmer: the programmer must provide correct policies for flows within the system at every level, which compounds as programs get larger, and as they have to perform more complex tasks, which makes writing modular programs difficult.
		
		While documentation is sparse, and the compiler's error messages often do not correspond to the true information flow problem, these merely exacerbate the core burden issue.
		
		In addition, though Jif has the advantage of being based on Java, interoperability is limited -- features like reflection do not exist, and external libraries require JIF signatures for their interfaces.
		}
	
	
\end{frame}

\begin{frame}{LIFTy -- An Alternate Approach}
	In my research thus far, the complexity o
\end{frame}

\begin{frame}{Other Implementations}
	\begin{enumerate}
		\item Paragon
		\item LIFTy
		\item Dynamic ones
	\end{enumerate}
\end{frame}

\begin{frame}{General Applicability}
	Oracle Guidelines
	
	Untrusted code vs Untrusted data
	
	Actual exploits
\end{frame}

\begin{frame}{Overall Thoughts}
	`Heavy' programmer burden
	
	Not relevant for most security issues
	
	Potential for `lighter' solutions (e.g. LIFTy, static analysis tools)
\end{frame}
%\section{Project Plan / Milestones}

\subsection{Evaluation Criteria}

	\subsubsection{Resources}
	
	\subsubsection{Duration / Timeframe}

\subsection{Motivating Example: Java Implementation}

	\subsubsection{Resources}
	
	\subsubsection{Duration / Timeframe}

\subsection{Motivating Example: JIF Implementation}

	\subsubsection{Resources}
	
	\subsubsection{Duration / Timeframe}
	
\subsection{Perform Evaluation Write-up}

	\subsubsection{Resources}
	
	\subsubsection{Duration / Timeframe}
	
\subsection{Perform Evaluation Write-up}

\subsubsection{Resources}

\subsubsection{Duration / Timeframe}

\subsection{Extension Goals}

\subsection{Timeline}

\bibliographystyle{IEEETran}
\bibliography{bibliography}

\end{document}